\subsection*{1. }
فرض کنید $DFA$ای به نام $M$ داریم که زبان منظم $L\;\subseteq \Pi^*$ را تشخیص می‌دهد. حال باید یک $DFA$ دیگر طراحی کنیم تا $h^{-1}(L)$ را تشخیص بدهد.
\begin{center}
    $M\;=\;(Q,\Pi,\delta,q_0, F)$\\
    $M'\;=\;(Q', \Sigma, \delta', q'_0, F')$\\[0.1in]
\end{center}
حال باید $M'$ را طوری طراحی کنیم که پاسخگوی مسئله باشد. استیت‌های هر دو را یکسان در نظر می‌گیریم. $Q'\;=\;Q$.
\newline
استیت شروع و
استیت های نهایی در ماشین $M'$ دقیقا مشابه ماشین $M$ خواهد بود. $F'\;=\;F$ و 
$q'_0\;=\;q_0$.
\newline
برای $\delta'$، این منطق را پیش می‌گیریم که به ازای استیت دلخواه $s$ و ورودی $t$ از الفبای این ماشین، به استیتی خواهیم رفت که ماشین $M$، با شروع از استیت $s$ و مشاهده $h(t)$ به آن خواهد رفت.
\newline
بنابراین ماشین $M'$ را تعریف کردیم. حال برای اثبات درستی آن داریم:
\newline
رشته $n\;=\;n_1n_2\ldots n_k$ را در نظر بگیرید.
ماشین $M$ با دیدن $h(n_1n_2\ldots n_k)$ به استیتی می‌رسد که ماشین $M'$ با دیدن خود رشته $n$ به آن می‌رسد. در این حال اگر $M$ به استیت نهایی نرسد، یعنی $n\notin h^{-1}(L)$. اما اگر به استیت نهایی برسد در نتیجه $h(n)\in L$ و $n\; \in \; h^{-1}(L)$
خواهد بود. که $M'$ نیز دقیقا کار درست و مورد انتظار را انجام می‌دهد.\newline
در نتیجه درستی $M'$ نیز اثبات شد.

\subsection*{2. }
الفبای $\Pi$ و هم‌ریختی‌های $g,\;h,\;k$ را به شکل زیر تعریف می‌کنیم:
\begin{center}
    $\Pi \;=\; \{x,\bar{x}\;|\; x \in \Sigma\}$
\end{center}
\begin{center}
    $g(a) = \begin{cases}
            x,\;& a=x\\
            \epsilon, \;& a=\bar{x}
        \end{cases}$\\[0.1in]
\end{center}
\begin{center}
    $k(a) = \begin{cases}
            \epsilon,\;& a=x\\
            x, \;& a=\bar{x}
        \end{cases}$\\[0.1in]
\end{center}
\begin{center}
    $h(a) = \begin{cases}
            x,\;& a=x\\
            x, \;& a=\bar{x}
        \end{cases}$\\[0.1in]
\end{center}
طبق تعریف، $h$ یک هم‌ریختی بسیار خوب و دوتای دیگر، دو هم‌ریختی خوب هستند.
حال داریم:
\begin{itemize}
    \item 
    $k^{-1}(B)$ شامل رشته‌هایی است که قرار دادن حروف با بار آنها، رشته‌ای در $B$ می‌سازد.
    \item 
    $g^{-1}(A)$ شامل رشته‌هایی است که کنار هم قرار دادن حروف بدون بار آنها، رشته‌ای در $A$ می‌سازد.
\end{itemize}
بنابراین اشتراک این دو، شامل رشته‌هایی می‌شود که حروف بدون بار آن کنار هم رشته‌ای از $A$ و حروف با بار آن رشته ای از $B$ می‌سازند. بنابراین اشتراک این دو همان شافل است که روی حروف مربوط به $B$ بار گذاشته شده است. هم‌ریختی $h$ هم علامت بار را از روی حرف برمی‌دارد. درنتیجه داریم:
\\[0.15in]
\begin{center}
    $h(g^{-1}(A) \cap k^{-1}(B)) =\; \text{Shuffle}(A,B)$
\end{center}

