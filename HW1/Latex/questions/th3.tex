\textbf{پرسش سوم}\\[0.1in]
\begin{enumerate}
    \item[1. ]
بگمارید که 
\lr{A = \{0, 30, 45, 60, 90, 120, 135, 150, 180, 210, 225,
240, 270, 300, 315, 330, 360\}}
رابطه‌ی $R$ روی $A \times A$ به گونه‌ی زیر تعریف شده است.\\
\begin{center}
    $(a, b)R(c, d) \Leftrightarrow \sin a \cos b = \sin c \cos d$
\end{center}
\begin{enumerate}
    \item[(آ)]
     ‌بررسی كنيد كه آيا اين رابطه هم‌ارزی هست و سپس برای ادعای خود برهان بياوريد. 
     \\[0.02in]
    \item[(ب)]
     رده هم‌ارزی 
     $[(30, 60)]$
     را بنويسيد.\\[0.1in]
\end{enumerate}
\item[2.]
نشان دهید که رابطه $R$ 
روی مجموعه $A$
پادتقارنی است اگر و فقط اگر $R\cap R^{-1}$
زيرمجموعه‌ای از رابطه قطری
$\Delta = \{(a, a)\; | \; a \in A \}$
باشد.
\end{enumerate}