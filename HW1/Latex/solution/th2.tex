\\[0.1in]
هر دو طرف را اثبات می‌نماییم.\\[0.1in]
\textbf{بخش اول:}\\[0.07in]
می‌دانیم:
\begin{center}
    $x = x_1 \ldots x_n$\\
    $y = y_1 \ldots y_n$\\
    $x^R = x_n \ldots x_1$\\
    $y^R = y_n \ldots y_1$
\end{center}
فرض می‌کنیم که
$x \in K^{-1}L$.
حال در بخش اول کافی است اثبات نماییم:
$x \in (L^R(K^R)^{-1})^R$. داریم:
\begin{center}
    $if \; x \in K^{-1}L \; then \; yx \in L, \; 
    y \in K$\\[0.07in]
    $\longrightarrow (yx)^R \in L^R \Longrightarrow
    x^Ry^R \in L^R, \; \; y^R \in K^R$
\end{center}
بنابراین طبق تعاریف داریم:
\begin{center}
    $x^R \in L^R(K^R)^{-1}$
\end{center}
از طرفی می‌دانیم که 
$(x^R)^R = x$،
پس:
\begin{center}
    $x \in (L^R(K^R)^{-1})^R$
\end{center}
بنابراین سمت اول تساوی اثبات شد و طرف چپ زیرمجموعه طرف راست می‌باشد.\\[0.2in]
\textbf{بخش دوم:}\\[0.07in]
دقیقا برعکس بخش اول می‌توان عمل نمود. داریم:
\begin{center}
    $if \; x \in (L^R(K^R)^{-1})^R \; \; then \; \; x^R \in L^R(K^R)^{-1}$
\end{center}
طبق تعاریف $y^R$ای داریم که:
\begin{center}
    $y^R \in K^R \; \; and \; \; x^Ry^R \in L^R$\\[0.07in]
    $\longrightarrow (yx)^R \in L^R \longrightarrow
    yx \in L,\; \; y \in K$
\end{center}
همان‌طور که مشخص می‌باشد طبق تعریف داریم:
\begin{center}
    $x \in K^{-1}L$
\end{center}
بنابراین اثبات نمودیم هر دو زیرمجموعه دیگری و در نتیجه برابر می‌باشند.\\[0.2in]