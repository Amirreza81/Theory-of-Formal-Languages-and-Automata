\subsection*{1. $L \;\in\; DTIME(n)$}
زبان L منظم است پس یک DFA دارد.
حال ماشین تورینگ M را برای این زبان میسازیم.
به نوعی که با خواندن ورودی،
هد به راست رفته و روی استیت های DFA جا به جا می‌شود.
بعد از پایان ورودی، اگر در
\lr{accepting state}
بودیم، آن را اکسپت و در غیر این صورت آن را رجکت می‌نماییم.
به دلیل اینکه ماشین M تنها یک بار روی رشته ورودی حرکت کرده است، پیچیدگی زمانی برابر 
$O(n)$ است و 
$L\;\in\;DTIME(n)$.

\begin{figure}[H]
    \centering
    \includegraphics[scale=0.67]{solution/2-1.png}
\end{figure}

\subsection*{2. $L \;\in\; DTIME(n^3)$}
حال در نظر می‌گیریم که L مستقل از متن می‌باشد.
از دروس قبل الگوریتم $cyk$ را به یاد داریم.\\
فرم نرمال چامسکی 
(CNF)
زبان L را در نظر بگیرید.
( که می‌دانیم در زمان چندجمله‌ای حاصل می‌شود. )
\\
حال برای اجرای الگوریتم CYK،
یک ماشین تورینگ دونواره را در نظر بگیرید.
یک نوار را برای خواندن ورودی اختصاص می‌دهیم.
از نوار دیگر برای نگه داشتن 
\lr{dynamic programming state}
استفاده می‌کنیم.
نوار دوم را در واقع با ترتیب 
lexicographic 
مجموعه ها قرار گرفته اند و با $\#$
جدا شده‌اند.
برای هر متغیر نیز یک خانه در تمام مجموعه ها در نظر میگیریم و آن خانه را در صورت برابری، 1 می‌کنیم.
که در این حالت هر مجموعه تعداد
$|V|$ خانه 
روی نوار دوم اشغال می‌کند.
البته این موضوع را در همان اسلایدها هم بررسی کردیم و نشان دادیم که این الگوریتم، پیچیدگی زمانی برابر 
$O(n^3)$
دارد. در نتیجه 
$L \;\in\; DTIME(n^3)$.

\subsection*{3. $L \;\in\; NTIME(n)$}
ابتدا به تصویر زیر توجه کنید.
\begin{figure}[H]
    \centering
    \includegraphics[scale=0.65]{solution/2-2.png}
\end{figure}

حال همان فرم نرمال چامسکی یا CNF را در نظر بگیرید.
می‌دانیم برای هر رشته w عضو این زبان،
یک derivation به حداکثر طول
$2|w| - 1$ وجود دارد.
\lr{($|w| - 1$ adding variable and $|w|$ converting to terminals.
)}
حال ماشین تورینگ غیرقطعی M را برای زبان مورد نظر میسازیم که به طور غیر قطعی عمل میکند.
طبق تصویر بالا و نکته ذکر شده داریم:
\begin{center}
    $O(2|W|) = O(|w|) = O(n) \; \Longrightarrow \; L \in NTIME(n)$.
\end{center}