برای حل این سوال از لینک‌های 
\href{https://datamove.imag.fr/denis.trystram/SupportsDeCours/TuringMachine2Basis.pdf}{یک} و 
\href{https://cs.uwaterloo.ca/~s4bendav/files/CS360S21Lec14.pdf}{دو}
کمک می‌گیریم.

\subsection*{1. }
برای این بخش الفبای روی 
tape 
که همان $\Gamma$ است را به 0 و 1 انکود می‌کنیم. با استفاده از 
$\Delta$ بین انها فاصله میگذاریم.
حال برای انکود کردن هر یک از الفبای ورودی tape
ماشین تورینگ به ماشین تورینگ استاندارد به اندازه هر سیمبل در تورینگ ماشین با یک رشته به طول
$q = \log_2(m)$ 
انکود می‌شود.بنابراین برای نوشتن و خواندن روی tape در ماشین استاندارد
q، برابر پیچیدگی در ماشین قبلی را داریم.
پس پیچیدگی در ماشین استاندارد برابر است با:
$qT(n) = O(qT(n)) = p(T(n))$.\\
که تبدیل گفته شده به دست آمد.\\
در واقع به طور خلاصه و شفاف‌تر، 
اگر 
$|\Gamma| = n$، هر کاراکتر را با
$\log_2(n)$ بیت کد میکنیم.
بین هر دو کاراکتر نیز $\Delta$ می‌گذاریم.
برای مشخص کردن پایان رشته نیز 2تا $\Delta$ می‌گذاریم.
حال برای شبیه‌سازی، هر حرکت راست یا چپ در تورینگ اول، معادل 
$\log_2(n)$ بیت حرکت راست یا چپ خواهد بود.
همچنین برای تغییر یک کاراکتر روی tape،
معادل تغییر دادن همه $\log_2(n)$ بیت می‌باشد.\\
در نتیجه اگر پیچیدگی ماشین اولیه برابر $T(n)$ باشد، پیچیدگی ماشین استاندارد برابر
$O(\log_2(n)T(n))$ است که کارا بودن اثبات می‌شود.

\subsection*{2. }
برای این بخش، طبق اسلایدها می‌دانیم هر ماشین تورینگ 2 نواره را میتوان در زمان چندجمله‌ای به ماشین تورینگ تک نواره معادل تبدیل کرد. حال یک ماشین تورینگ تک نواره مانند بخش 1 خواهیم داشت. بنابراین کاراست.\\
برای توضیح درست‌ و کامل‌تر داریم:
\\
میدانیم که برای هر ماشین تورینگ دو نواره با پیچیدگی زمانی
$T(n)$، ماشین تورینگ تک‌نواره با پیچیدگی زمانی 
$O(T^2(n))$ وجود دارد که معادل هستند.
بنابراین ابتدا ماشین دو نواره را به ماشین تک نواره تبدیل میکنیم
.
تنها ممکن است الفبا لزوما همان الفبای گفته شده نباشد. حال از این بخش به بعد مانند بخش اول این سوال عمل می‌کنیم و کارا بودن اثبات می‌شود.

\subsection*{3. }
طبق اسلایدهای درس می‌دانیم هر ماشین تورینگ با 2 سر را می‌توان به 
ماشین تورینگ 2نواره معادل در زمان چندجمله‌ای تبدیل کرد.
یک کپی از ورودی میگیریم و ان را در نوار دوم قرار میدهیم.
باید محتویات این دو نوار را یکسان نگه داریم.
اگر سر نوار اول در خانه k بود، در صورت تغییر خانه،
سر نوار دوم به k میرود و بعد انجام تغییر به جایگاه قبلی باز خواهد گشت.
( برای هر دو نوار برقرار است. )
\\
هر عملیات تغییر خانه توسط یکی از سرهای ماشین اولیه،
$O(T(n))$ زمان خواهد برد. در نتیجه پیچیدگی زمانی ماشین دو نواره برابر 
$O(T^2(n))$ خواهد شدو تبدیل چندجمله‌ای است.
حال از اینجا به بعد مطابق بخش قبل عمل خواهیم کرد.
کارا بودن این بخش هم اثبات می‌شود.
\\