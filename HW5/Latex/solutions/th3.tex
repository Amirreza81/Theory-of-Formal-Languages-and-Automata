برای حل این سوال از این 
\href{http://www.cs.mun.ca/~kol/courses/3719-w12/pnp_notes.pdf}{لینک}
کمک گرفته شده است.\newline
به ترتیب دو طرف را اثبات می‌نماییم:
\subsubsection*{طرف اول: }
ماشین تورینگ غیرقطعی M برای زبان L را در نظر بگیرید.
حال 
\lr{maximum branching factor} را برابر 
b در نظر بگیرید.
از آنجایی که تعداد استیت‌ها و الفبا محدود است، این مقدار نیز محدود است.
همچنین فرض کنید پیچیدگی زمانی M برابر
$f(m)$ باشد. برای رشته $s_1$ که توسط ماشین بیان شده پذیرفته می‌شود، رشته $s_2$ را 
جوری در نظر میگیریم و تعریف می‌کنیم که هر کاراکتر مشخص می‌کند که در راس
فعلی درخت computation،
به کدام راس در لایه بعدی برویم.
چون رشته $s_2$ مسیری را مشخص می‌کند که M با طی کردن آن مسیر برای ورودی $s_1$، آن را تشخیص داده و می‌پذیرد و به accept میرود.
در نتیجه برای هر رشته $s_1$ یک رشته $s_2$ وجود دارد که طول رشته $s_2$ حداکثر برابر 
$f(|s_1|)$ می‌باشد.
\\
در نتیجه مطابق نوتیشن صورت سوال، برای هر شته $x$ رشته 
$y \;\in\; \Sigma^{p(|x|)}$ وجود دارد و آن را در R قرار می‌دهیم. بنابراین:
\begin{center}
    $L = \{ x\;\in\;\Sigma^*|\;\exists y\;\in \Sigma^{p(|x|)}, \;(x,y)\;\in\;R\}$
\end{center}
در نتیجه از تعریف اول به تعریف دوم رسیدیم 
( فراموش نکنید که زبان L در رده NP است. )

\subsubsection*{طرف دوم: }
چندجمله‌ای $p$ و رابطه R را داریم.
برای اثبات برابری این تعریف با تعریف اول، یک ماشین تورینگ غیرقطعی را به شکلی میسازیم که رشته به طول 
$p(|x|)$ را حدس بزند و بعد از آن بررسی نماید که آیا 
$(x,y)$ عضو R هستند یا خیر
حال در این صورت اگر حدس، درست بود قبول می‌شود و اگر نبود، رد خواهد شد.\\
در نتیجه ماشین تورینگ ذکر شده، در زمان چندجمله‌ای، زبان گفته شده را تشخیص خواهد داد.\\
در نتیجه از تعریف دوم به تعریف اول رسیدیم.
\\
در واقع توجه داتشه باشید ماشین تورینگ چندنواره غیرقطعی ساختیم که میدانیم با تک نواره معادل است.\\
برای تکمیل بحث و این سوال، به توضیحات زیر توجه فرمایید:

\begin{figure}[H]
    \centering
    \includegraphics[scale = 0.6]{solution/3-2.png}
    \includegraphics[scale = 0.6]{solution/3-1.png}
\end{figure}